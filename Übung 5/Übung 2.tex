\documentclass{article}
\usepackage[utf8]{inputenc}
\usepackage[ngerman]{babel}
\usepackage{floatrow}
%\usepackage[singlelinecheck=off]{caption}
\author{Christin Berger}
\title{CS102\ \LaTeX\ Übung}
\date{\today}
\begin{document}
\maketitle
\section{Das ist der erste Abschnitt}
Hier könnte Ihre Werbung stehen :)
\section{Tabelle}
Unsere wichtigsten Daten finden Sie in Tabelle 1.
\begin{table}[h]
\begin{tabular}{c|c|c|c}
\quad & Punkte erhalten & Punkte möglich & \% \\
\hline
Aufgabe 1 & 2 & 4 & 0.5\\
Aufgabe 2 & 3 & 3 & 1\\
Aufgabe 3 & 3 & 3 & 1\\
\end{tabular}
\caption{Diese Tabelle kann auch andere Werte beinhalten}
\end{table}
\label{Tab 1}
\section{Formeln}
\subsection{Pythagoras}
Der Satz des Pythagoras errechnet sich wie folgt: $a^{2}+b^{2}=c^{2}.$ Daraus können wir die Länge der Hypothenuse wie folgt berechnen: $c=\sqrt{a^{2}+b^{2}}.$
\subsection{Summen}
\begin{flushleft}
Wir können auch die Formel für eine Summe angeben:\\
\end{flushleft}
\begin{equation}
\centering
s=\sum_{i=1}^n{i}=\frac{n*(n+1)}{2}
\end{equation}
\end{document}



